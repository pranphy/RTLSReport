\section*{Preface}
Nepal Government has introduced a new trial system for providing license to drivers. Traffic authorities present at the trial venue decide on the basis of inspection whether the driver is qualified to acquire a driving license or not.

Use of technology in the field of road traffic management has long prevalence in Nepal, specially though the use of automatic traffic lights placed at almost every chowks of the cities with busy traffic. Use of breathalyzers to control the drink-drive cases in Nepal has had a great positive impact reducing significantly the road accidents. This has eased the task of policeman as well as increased human safety. 


Also the use of radar gun to measure the vehicle speed on road has assisted the police greatly instantly capturing a overspeed vehicle thereby preventing a potential accident. 


Use of technology has clearly have a positive impact in the traffic management area. One area that still lacks proper use of technology is traffic license acquiring process by riders and drivers. So far in Nepal the rider is let drive on the specially designed track to decide whether he is capable enough to drive safely in the busy road. Human invigilator inspects the whole process.But human eye inspection is not reliable enough to decide, with great accuracy the prevalent license trial system.


Articles in newspapers suggest that the increased road accidents are directly related to the drivers driving their vehicle without acquiring license or those acquiring fraud license without passing the trial examination.


Though the new trial system improves the standard of examination and tests the drivers properly, rumors of people acquiring fraud driving license do not cease to exist. Thus the idea is to build a system where a computer rather than human eye decides the validity of a trial, thus eliminating the chances of fraud license.


